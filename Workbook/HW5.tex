%=======================02-713 LaTeX template, following the 15-210 template==================
%
% You don't need to use LaTeX or this template, but you must turn your homework in as
% a typeset PDF somehow.
%
% How to use:
%    1. Update your information in section "A" below
%    2. Write your answers in section "B" below. Precede answers for all 
%       parts of a question with the command "\question{n}{desc}" where n is
%       the question number and "desc" is a short, one-line description of 
%       the problem. There is no need to restate the problem.
%    3. If a question has multiple parts, precede the answer to part x with the
%       command "\part{x}".
%    4. If a problem asks you to design an algorithm, use the commands
%       \algorithm, \correctness, \runtime to precede your discussion of the 
%       description of the algorithm, its correctness, and its running time, respectively.
%    5. You can include graphics by using the command \includegraphics{FILENAME}
%
\documentclass[11pt]{article}
\usepackage{amsmath,amssymb,amsthm}
\usepackage[noend]{algpseudocode}
\usepackage{graphicx}
\usepackage[margin=1in]{geometry}
\usepackage{mathtools}
\usepackage{fancyhdr}
\usepackage{algorithmicx}
\setlength{\parindent}{0pt}
\setlength{\parskip}{5pt plus 1pt}
\setlength{\headheight}{13.6pt}
\newcommand\question[2]{\vspace{.25in}\hrule\textbf{#1: #2}\vspace{.5em}\hrule\vspace{.10in}}
\renewcommand\part[1]{\vspace{.10in}\textbf{(#1)}}
\newcommand\analysis{\vspace{.10in}\textbf{Analysis: }}
\newcommand\algorithm{\vspace{.10in}\textbf{Algorithm: }}
\newcommand\correctness{\vspace{.10in}\textbf{Correctness: }}
\newcommand\runtime{\vspace{.10in}\textbf{Running time: }}
\pagestyle{fancyplain}
\lhead{\textbf{\NAME\ (\ANDREWID)}}
\chead{\textbf{CS180 HW\HWNUM}}
\rhead{02-713, \today}
\begin{document}\raggedright
%Section A==============Change the values below to match your information==================
\newcommand\NAME{Zeyuan Xu}  % your name
\newcommand\ANDREWID{004255573}     % your andrew id
\newcommand\HWNUM{5}              % the homework number
%Section B==============Put your answers to the questions below here=======================

% no need to restate the problem --- the graders know which problem is which,
% but replacing "The First Problem" with a short phrase will help you remember
% which problem this is when you read over your homeworks to study.

\question{1}{Printing Paragraph} 

\analysis:
Let
\begin{math}
s(i,j) = M - j + i - \displaystyle\sum_{k=i}^{j} l_k
\end{math}

be the number of extra spaces required to put words from i to j on the same line. 
When s(i,j) is greater or equal to 0, we can fit words into the line. Otherwise, we cannot do that.
Therefore, we can summurize the cost c(i,j) as:

\begin{math} 
c(i,j)=
\begin{cases}
\infty, & \text{if}\ s(i,j) < $0$ \\
0, & \text{if}\ s{i,j} \geq $0$ and j = n \\
s(i,j)^3, & \text{otherwise} 
\end{cases}
\end{math}

The dynamic programming method:
Let C(j) be the optimal method for printing n words from word 1 to word j,  Then we have \newline
\begin{math}
C(0) = 0  \newline
C(j) = \min_{j \geq i \geq 0} C(i)+ C(i,j)    \hspace{5mm} for  \hspace{5mm} j > 0 
\end{math}\newline 
for our algorithm, the input is an integer array l[ ], with  slot i representing the length of word i
M is the maximum length per line, and n is the size of our input array.

\runtime:
The complexity of the algorithm is $O(n^2)$
The space requirement for the algorithm is about $O(n)$ complexity 

\algorithm:
\begin{algorithmic}
\State $EX[n+1][n+1]$ \algorithmiccomment{number of extra spaces on a single line for word i to j}
\State $LC[n+1][n+1]$ \algorithmiccomment{the cost for a single line}
\State $C[n+1]$ \algorithmiccomment{the total cost of our optimal solution for word from 1 to j}
\State $p[n+1]$ \algorithmiccomment{used to print the optimal solution}
\State $i$ and $j$

\algorithmiccomment{calculate the extra space for each word when they are put into a single line}
\For{\texttt{i from $1$ to $n$}}
     \State \texttt{$EX[i][i]$ = $M$ - $l[i-1]$}
     \For{\texttt{$j$ from $i+1$ to $n$}}
	\State \texttt{$EX[i][j]$ = $EX[i][j-1]$ - $l[j-1]$ - $1$} 
     \EndFor
\EndFor

\algorithmiccomment{calculate the corresponding line costs}
\For{\texttt{$i$ from $1$ to $n$}}
	\For{\texttt{$j$ from $1$ to $n$}}
		\If{$EX[i][j]$ < $0$}
			\State \texttt{$LC[i][j]$ = infinity}
		\ElsIf{ j = n and  EX[i][j] greater or equal to 0}
			\State \texttt{$LC[i][j]$ = $0$}
		\Else
			\State \texttt{$LC[i][j]$ = $EX^3[i][j] $}
		\EndIf
	\EndFor
\EndFor

\algorithmiccomment{calculate minimum cost}
\State \texttt{$C[0]$ = $0$}
\For{\texttt{$j$ from $1$ to $n$}}
	\State \texttt {$C[j]$ = infinity}
	\For{\texttt{$i$ from $1$ to $j$}}
		\If{\texttt{C[i-1] != infinity and LC[i][j] != infinity and C[i][j] + LC[i][j] < C[j]}}
			\State \texttt{$C[j]$ = $C[i-1]$ + $LC[i-1]$}
			\State \texttt{$P[j]$ = $i$}
		\EndIf
	\EndFor
\EndFor
\end{algorithmic}


\question{2}{Palindrome subsequence}

\analysis: 
approach the problem using bottom-up approach, by beginning with subsequences with length 1 all the way up to the string with length equal to the input string length. If the input string has length n, The data structure used is a n by n matrix. 

Suppose that the input is a string s with length n 

\runtime:
The Time complexity of the algorithm is $O(n^2)$


\algorithm:
\begin{algorithmic}
\State $chars[n] \gets $s \algorithmiccomment{convert the input string to a char array}
\State $R\gets $R[n][n] \algorithmiccomment {Initialize a new n by n matrix}
\If {$n = 0$}
    \State \textbf{return} $0$ \algorithmiccomment {empty string}
\Else
      \For{\texttt{i from $1$ to $n$}}
        \State \texttt{R[i][i] = $1$}     \algorithmiccomment{add characters to the matrix column}
      \EndFor
      \For{\texttt{sublen from $2$ to $n$}}
	\For{\texttt{i from $0$ to $n$-$sublen$}}
      	 \State \texttt{$j$ = $i$ + $sublen$ - $1$}
	 \If{ chars[i] = chars[j] and $sublen$ = $2$ }
	 	\State \texttt{R[i][j] = 2}
	 \ElsIf{chars[i] = chars[j]}
		\State \texttt{R[i][j] = R[i+1][j-1] + $2$}
	 \Else
		\State \texttt{R[i][j] = max(R[i+1][j] and LP[i][j-1])}
	 \EndIf
	\EndFor      
      \EndFor
\State \texttt{return R[0][n-1]}
\EndIf
\end{algorithmic}

\question{3}{Minimum Cost of Cutting a String}

\analysis:
the input of the problem is a stirng s with length n, and a set of m indices to break the string according to the indexes 
we need a best ordering of the breaks to minimize the cost of cutting the string 
pick index i and j,  where i is less than j, and i, j are between 0 and n 
denote the minimum cost for cutting the substring from index i to j as C(i,j)
Recurrence: C(i,j) = mininum of  (length of substring + C(i,k) + C(k,j) where k is between i and j)
Build a n by n table and find the minimum cost of each slot (i,j) where i and j are between 0 and n 

\runtime:
building up the table takes $O(n^2)$
at most we need to look up n items and compare, which gives $O(n)$
total cost is about $O(n^3)$

\algorithm:
\begin{algorithmic}
\State $C[n][n]$ \algorithmiccomment{the cost table that stores cost for substring(i,j)}

\end{algorithmic}


\question{4}{Schedule to minimize the lateness}
Part(a) : Suppose that we have an optional solution J, and suppose we have an inversion in the optimal solution:
if job i is scheduled before job j but \begin{math} d_i > d_j \end{math}, in this case, the finish time will still be before \begin{math}d_i \end{math}
and the swapping will not change finish time for other jobs, this is still optimal after the swapping.

Any permutation of the optimal solution is a combination of swappings in the optimal soliution. Since swapping doesn't change 
optimality, we can have an optimal solution where the dealines are in increasing order. 

Part(b): 
Suppose that $d_1\le d_2\le...\le d_n=D$, 
since in part (a), we have proven that there is an optimal solution in increasing order. 
so we recognize subproblems as $OPT(i,j)$, which means the maximal number of schedulable jobs among $1, 2, ..., i$ under the constraint that all the chosen job must finish by time $j$. Then $OPT(i,j)$ is achieved either by including job $i$ in the schedule (at the end) or not, i.e.,
\begin{align}
OPT(i,j) = \max{\{OPT(i-1,j-t_i)+1, OPT(i-1,j)\}}
\end{align}
where $i\ge 1$ and $j\ge 0$, and if $j < t_i$, job $i$ cannot be involved in the schedule, so $OPT(i,j) = OPT(i-1,j)$ in this case. And the base case is
\begin{align}
OPT[0][j] = 0 \textrm{ for } j=0, 1, ..., D
\end{align}
Apply dynamic programming, we use a 2D array \texttt{OPT[0..n][0..D]} to store the values of $OPT(i,j)$
In the end, $OPT[n][D]$ will be our final solution.

\algorithm:
\begin{algorithmic}
\State $OPT[n][D]$
\For{\texttt{i from 0 to D}}
	\State \texttt{OPT[0][j] = 0}
\EndFor
\For{\texttt{i from 1 to n}}
	\For{\texttt{j from 1 to D}}
		\If{\texttt{$t_i$< $j$}}
			\State \texttt{$OPT[i][j]$ = $OPT[i-1][j]$}
		\Else 
			\State \texttt{ max(OPT[i-1][j-\begin{math}t_i\end{math}] and OPT[i-1][j])}
		\EndIf
	\EndFor
\EndFor
\end{algorithmic}

\runtime:
after we have constructed the table, we need backtracking to find the value  $OPT[n][D]$,if $OPT[i][j]$ = $OPT[i-1][j]$,  job $i$ is not included in the schedule,
if OPT[i][j] = OPT[i-1][j-\begin{math}t_i\end{math}]+$1$, job $i$ is included in the schedule. Do this backward tracing until we reach$OPT[0][0]$ 
sorting the elements takes $O(n\log n)$, filling table takes constant time $O(D)$. backtracking takes $O(n)$
thus we need $O(nD)$



\end{document}











